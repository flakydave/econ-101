\documentclass[xcolor={dvipsnames},pdf, hyperref={colorlinks=true, citecolor=ForestGreen, linkcolor=BlueViolet, urlcolor=Magenta}]{beamer}
\usetheme{Frankfurt}  
\usecolortheme{whale}
\usepackage{tikz} 
\usepackage{graphicx}
\usepackage{dsfont}
\usepackage{hyperref}
\usepackage{alltt}
\usepackage{enumerate}
\usepackage{amsthm}
\theoremstyle{definition}
\newtheorem{exmp}{Example}[section]
\usepackage{verbatim}               % useful for \begin{comment} and \end{comment}
\usepackage{eurosym}                % used for euro symbol
\usepackage{caption} 
\usepackage{graphicx}
\usepackage{adjustbox}
\graphicspath{{Figures/}}
\usepackage{subcaption}
\usepackage{color}
\usepackage{float}
\usepackage{amssymb}
\usepackage{sgamevar}
\usepackage{remreset}% tiny package containing just the \@removefromreset command
\makeatletter
\@removefromreset{subsection}{section}
\makeatother
\setcounter{subsection}{1}


\newcommand{\defn}[1]{\textbf{#1}}


%Instructor version
\newcommand{\blank}[0]{}
\newcommand{\ddp}[1]{{\textcolor{ForestGreen}{#1}}} 
\newcommand{\dd}[1]{{\underline{\textcolor{ForestGreen}{#1}}}}

%Student version
%\newcommand{\blank}[0]{\vspace{2em}}
%\newcommand{\dd}[1]{\underline{\hspace{3cm}}} 
%\newcommand{\ddp}[1]{}

\addtobeamertemplate{navigation symbols}{}{%
	\usebeamerfont{footline}%
	\usebeamercolor[fg]{footline}%
	\hspace{1em}%
	\insertframenumber/\inserttotalframenumber
}

\section{Defining GDP}

%% preamble
\title{Measuring a Nation's Income}
\author{David A. D\'iaz}
\institute{UNC Chapel Hill}
\date{}

\AtBeginSection[] %Section links on slides

\begin{document} 
	
	\begin{frame}
		
		\titlepage
		
	\end{frame}



\begin{frame}{GDP}
\begin{itemize}
	\item This section will look at how we measure the total income of a nation -- its gross domestic product. 
	\item GDP measures both the total \dd{income} of everyone in the economy and the total \dd{expenditure} on the economy's output of goods and services. 
	\item That is, for the economy as whole, \dd{income} must equal \dd{expenditures}. 
	\item \defn{GDP:} The market value of all \underline{final} goods and services produced \underline{within} a country during a specific period of time (typically a year).
\end{itemize}
\end{frame}

\begin{frame}{GDP}
\begin{itemize}
	\item GDP adds together many different kinds of products into a single measure of economic activity. To do so, it uses \dd{prices}, which reflect the value of goods.
	\item Includes all items produced in the economy and sold \textit{legally} in markets.
	\begin{enumerate}[i.]
		\item Includes market value of housing services. This includes the rental value of owner-owned housing.
		\item Excludes items produced and sold illicitly and most items that are produced and consumed within a household. (e.g., prostitution, gambling, tomatoes grown at home)
	\end{enumerate}
\end{itemize}
\end{frame}

\begin{frame}{GDP}
\begin{itemize}
	\item GDP \textbf{only} includes the value of final goods. The value of intermediate goods used to produce final goods is not included. 
	\item An exception to this rule is when an intermediate good is added to a firm's inventory for use or sale at a later date: Additions to inventory \dd{increase} GDP and when the goods are later used or sold, the reductions in inventory \dd{detract} from GDP.
\end{itemize}
\end{frame}

\begin{frame}{GDP}
\begin{itemize}
	\item Includes goods and services that are \dd{currently} produced. The sale of a used good is \textbf{not} included. 
	\item Items are included in a nation's GDP only if they are produced \dd{domestically}. The nationality of the producer does not matter.
	\item GDP is usually reported at intervals of a year or a quarter of a year.
\end{itemize}
\end{frame}

\begin{frame}{GDP}
\begin{exmp} 

	For each of the following, find out how much each transaction contributes to GDP.
	\begin{enumerate}
		\item A metal company sells steel to a bicycle company for \$150. The bicycle company uses this steel to produce a bike, which it sells for \$250. 
		\item Catalina owns two houses. She rents one house to the Munsters for \$15,000 per year. She lives in another house, which she could rent for \$18,000 a year.
	\end{enumerate}
\end{exmp}

\ddp{\pause (1) \$250. Only sale of final good adds to GDP.\\
\pause (2) \$33,000. GDP includes market value of housing services and the rental value of owner-owned housing.}
\end{frame}

\begin{frame}{GDP}
\small
\begin{exmp}
		For each of the following, find out how much each transaction contributes to GDP.
	\begin{enumerate}
		\item The Suarez family bought a newly constructed home in 2009 for \$300,000. In 2016, they sold the house for \$345,000. How much did the sale of this home in 2016 contribute to 2009 GDP? To 2016 GDP?

		\item Stella's Blender company had \$50,000 of blenders in inventory at the end of 2009. In 2010, it sold \$300,000 of blenders to consumers and had \$40,000 in inventory at the end of the year. How much did the blenders produced by the company add to GDP in 2010?
	\end{enumerate}
\end{exmp}

\ddp{\pause (1) \$0 in both years. The sale of used goods does not contribute to GDP. The original sale of the home contributed \$300,000 to GDP in 2009. \\
\pause (2) $\$300 + 40 - 50 = \$290$K. Reduce GDP by \$50K because inventory sold was produced in 2009. Remember we only want to include what was produced in 2010.}
\end{frame}

\section{The Expenditure Approach}

\begin{frame}{GDP}
\begin{itemize}
	\item We will decompose GDP into different types of spending. This way of analyzing GDP is called the \dd{expenditure approach}.
	\item The other main way of looking at GDP is the \dd{factor income approach}. It consists of (1) wages and salaries, (2) interest, (3) rent, and (4) profit. 
	\item We will not focus on this approach, but remember that regardless of which way you measure GDP, it will be (approximately) the same.
\end{itemize}
\end{frame}

\begin{frame}{GDP}

\begin{enumerate}
		\item \textbf{Consumption (C):} Spending by households on good and services. This excludes purchases of new housing!
		\item \textbf{Investment (I):} Spending on capital equipment, inventories, and structures. Includes household purchases of new housing.
		\item \textbf{Government purchases (G):} Spending on goods and services by local, state, and federal governments. Does not include transfer payments such as unemployment payments or welfare.
		\item \textbf{Net Exports (NX):} Spending on domestically produced goods by foreigners (exports) minus spending on foreign goods by domestic residents (imports).
	\end{enumerate}
\begin{itemize}
	\item Thus, we can write that $Y = $ \dd{$C + I + G + NX$}.
\end{itemize}
\end{frame}

\begin{frame}{GDP}
\begin{exmp}
	
	For each of the following examples, state what components of GDP are affected and if GDP will increase, decrease, or stay the same. 
	
	\begin{enumerate}

		\item	A metal company sells steel to a bicycle company for \$150. The bicycle company uses this steel to produce a bike, which it sells for \$250. 
		\item	The Suarez family buy a newly constructed home for \$300,000. 
		\item Natalie spends \$2,000 on a new laptop to use in her publishing business. The laptop was produced in China.
	\end{enumerate}
\end{exmp}

\ddp{\pause (1) $C$ and $Y$ increase by \$250. \\
\pause (2) $I$ and $Y$ increase by \$300,000. \\
\pause (3) $I$ increases by \$2,000, $NX$ decreases by \$2,000. $Y$ is unchanged.}
\end{frame}

\section{Nominal and Real GDP}

\begin{frame}{GDP}
\begin{itemize}
	\item If total spending increases from one period to another, it must be that at least one of two things occurred:
	\begin{enumerate}
		\item Output increased.
		\item Prices increased.
	\end{enumerate}
	\item A nation's productivity is measured by its output of goods and services. Thus, we need a way to disentangle the output effect from the price effect.
\end{itemize}
\end{frame}


\begin{frame}{GDP}
\begin{itemize}
	\item \defn{Nominal GDP:} The production of goods and services valued at \underline{current year prices}. 
	\[Y_t^N = \sum_i P_{it} \cdot Q_{it}\]
	\item 
	\defn{Real GDP:} The production of goods and services valued at \underline{constant prices} from a given base year. 
	\[Y_t^R = \sum_i P_{i,base} \cdot Q_{it}\]
	\item \defn{Real GDP per capita:} A measure of the average income of individuals in a country: $y = Y/N$.
\end{itemize}
\end{frame}


\begin{frame}{GDP}
\begin{exmp}
	\tiny
	Suppose a simple economy can only produce two goods: apples and bananas. The quantities produced in 2000, 2001, and 2002, as well as their prices, are shown in Table \ref{simp}.
	
	\begin{table}[h!]
		\centering
		\caption{Production in a Simple Economy}
		\label{simp}
		\begin{tabular}{c|c|c|c|c}       
			
			Year & Apples & $P_A$ & Oranges & $P_O$  \\
			\hline
			2000 & 1000 & \$1.50 & 2500 & \$2.00 \\
			2001 & 1005 & \$1.51 & 2555 & \$2.10 \\
			2002 & 1005 & \$1.52 & 2600 & \$2.45 \\
		\end{tabular}
	\end{table} 
	
	Calculate the nominal and real GDP for each year, using 2000 as the base year.

\end{exmp}
\tiny
\ddp{\pause Nominal GDP: \\
	\pause 2000: $1000 \times 1.50 + 2500 \times 2 = \$6,500$. \\ \pause 2001: $1005 \times 1.51 + 2555 \times 2.10 = \$6,883$. \\
	\pause 2002: $1005 \times 1.52 + 2600 \times 2.45 = \$7,898$. \\ 
	\pause Real GDP: \\
	\pause 2000: $1000 \times 1.50 + 2500 \times 2 = \$6,500$. \\ \pause 2001: $1005 \times 1.50 + 2555 \times 2 = \$6,618$. \\
	\pause 2002: $1005 \times 1.50 + 2600 \times 2 = \$6,708$.}
\end{frame}

\begin{frame}{GDP}
\begin{itemize}
	\item \defn{GDP Deflator:} A measure of the price level in a given year. 
	\[\text{GDP deflator$_t$} = (Y^N_t/Y^R_t) \times 100\]
	\item Essentially, the GDP deflator measures the current level of \dd{prices} relative to the level of prices in \dd{the base year}. Thus, it reflects only the changes in the prices of good and services.

	
\end{itemize}
\end{frame}

\begin{frame}{GDP}
\begin{itemize}
	\item We can use the GDP deflator to calculate the \dd{inflation rate} in an economy. This rate describes how fast the overall price level in an economy is rising (or falling). 
	\item Using the GDP deflator, the inflation rate is calculated as:
	
	\[\pi_{t+1} = (Deflator_{t+1} - Deflator_t)/Deflator_t \times 100\%\]
	
	
\end{itemize}
\end{frame}

\begin{frame}{GDP}
\begin{exmp} 
	Use Table \ref{simp} to calculate the GDP deflator for each year. What is the inflation rate from 2001 to 2002?
\end{exmp}
\ddp{\pause 2000: $(6500/6500)\times 100 = 100$. \\
	\pause 2001: $(6,883/6,618)\times 100 = 104$. \\
	\pause 2002: $(7,898/6,708)\times 100 = 118$. \\
	\pause $\pi_{2002} = (118-104)/104\times 100\% = 13.5\%$.}
\end{frame}

\section{GDP Growth}

\begin{frame}{GDP}
\begin{itemize}
	\item Often, we are more interested on the growth of GDP instead of its level (e.g., U.S. real GPD is higher than China's but Chinese GDP is growing at a faster rate)
	\item The growth rate of real GDP is computed as:
	 \[\hat{Y}_{t+1} = (Y_{t+1} - Y_t)/Y_t \times 100\%\]

\end{itemize}
\end{frame}

\begin{frame}{GDP}
\begin{itemize}

	\item If we are interested in the growth of the average individual's income, we'd want to find the growth rate of real GDP per capita:
	\[\hat{y}_{t+1} = (y_{t+1} - y_t)/y_t \times 100\%\]
	\item It turns out that since $y = Y/N$, we can write 
	\[\hat{y} \approx \hat{Y} - \hat{N}\]
	where $\hat{N}$ is the growth rate of the nation's population.
\end{itemize}
\end{frame}

\begin{frame}{GDP}

	\begin{exmp} 
		\scriptsize
		Suppose an economy had nominal GDP of \$1,000 and \$1,025 in 2014 and 2015, respectively. Additionally, real GDP in 2015 was \$1,014 (using 2014 as the base year). 
		\begin{enumerate}
			\item Calculate the growth rate of nominal and real GDP between 2014 and 2015. 
			\item The country had a population of 200 in 2014 and 210 in 2015. Calculate the growth rate of real GDP per capita.
		\end{enumerate}
	\end{exmp}
	\scriptsize
	\ddp{\pause Nominal GDP growth: (1025 - 1000)/1000 $\times 100$ = 2.5\%. \\
		\pause Real GDP growth: (1014-1000)/1000 $\times 100$ = 1.4\%. \\
		\pause Real GDP per capita growth: $\hat{N} = (210 - 200)/200 \times 100\% = 5\%.$ \\
		$y = Y/N \Rightarrow \hat{y}\approx \hat{Y} - \hat{N} = 1.4 - 5 = -3.6\%$. }
\end{frame}

\begin{frame}{GDP}
\begin{itemize}
	\item Because economic well-being is measured by real GDP, fluctuations in real GDP measure the overall state of the economy over time.
	\item \defn{Recession:} A period in which real GDP falls for consecutive periods (generally 2 or more).
	\item \defn{Expansion:} A period in which real GDP rises for consecutive periods.
\end{itemize}
\end{frame}



\begin{frame}{Readings and Assignments}
\begin{itemize}
	\item Today: Mankiw Ch. 23
	\item Next time: Mankiw 24
	\item Problem Set 4, section 2
\end{itemize}
\end{frame}

\end{document}